\section{О курсе}

Цель данного методического пособия состоит в изучении общих принципов работы компьютера и получении навыков работы с рядом пакетов. Студентам предлагается рассмотреть для получения базовых знаний и умений по работе с компьютером такие темы как: 
\begin{itemize}
\item основы теории информации;
\item сжатие компьютерных данных;
\item помехоустойчивое кодирование;
\item архитектура ЭВМ;
\item организация компьютерных сетей;
\item работа с офисными пакетами;
\item программное обеспечение профессионального программиста.
\end{itemize}

Наряду с этими темами, авторы этого пособия предоставляют возможность более детально ознакомиться с рядом современных пакетов, в их числе такие широко известные продукты Microsoft, как Microsoft Office Word и Microsoft Excel, и система компьютерной верстки TeX, имеющая свой собственный язык разметки. Освоение указанных пакетов позволит студентам получить полезные навыки по подготовке презентаций, научно-технических отчетов о результатах выполненной работы, в оформлении результатов исследований в виде статей и докладов на научно-технических конференциях. Также авторы пособия продемонсрируют не-которые методики использования программных средств для решения практических задач.