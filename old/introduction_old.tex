Цель данного методического пособия состоит в изучении общих принципов работы компьютера и получении навыков работы с рядом пакетов. Студентам предлагается рассмотреть для получения базовых знаний и умений по работе с компьютером такие темы как: 
\begin{itemize}
\item системы счисления;
\item сжатие данных;
\item помехоустойчивое кодирование;
\item архитектура ЭВМ.
\end{itemize}


Наряду с этими темами, авторы этого пособия предоставляют возможность более детально ознакомиться с рядом пакетов
, в их числе такие широко известные продукты Microsoft, как Microsoft Office Word и Microsoft Excel, и система компьютерной верстки TeX, имеющая свой собственный язык разметки.

Обозначенное выше подходит под определение самого курса "Информатика". Сам термин появился в 1957 году и носит определение: научная дисциплина, изучающая общие свойства и структуру информации, закономерности ее создания, преобразования, накопления, передачи и использования. Первое его упоминание в СССР произошло только в 1968 году, а отдельной наукой информатика стала только в 70ых годах двадцатого века. 
