\section{Представление отрицательных чисел в ЭВМ}

В электронных вычислительных машинах нет возможности обозначить знак <<минус>> перед числом. Существует несколько способов решения этой проблемы:
\begin{enumerate}
\item \textbf{Специальный знаковый бит} -- определенный бит означает знак числа.
\\\emph{Пример 1:}
\\$+5_{10} = 0101_{2}$, $-5_{10} = 1101_{2}$
\\В данном случае, знаковый бит -- старший.
\item \textbf{Фиксированное смещение} -- все числа уменьшены на какое--то определенное число.
\\\emph{Пример 2:}
\\$-5_{10} = 0000_{2}$, $-4_{10} = 0001_{2}$, \dots ,$+10_{10} = 1111_{2}$
\\В данном случае, все числа уменьшены на 5.
\item \textbf{Нега--двоичная система счисления} -- основание системы счисления равно $-2$.
\\\emph{Пример 3:}
\\$-4_{10} = 1100_{-2}$, $+5_{10} = 0101_{-2}$
\item \textbf{Обратный (инверсный) код} -- инвертируются все биты.
\\\emph{Пример 4:}
\\$+5_{10} = 0101_{2}$, $-5_{10} = 1010_{2}$
\item \textbf{Дополнительный код} -- инверсия всех бит плюс единица.
\\\emph{Пример 5:}
\\$+5_{10} = 0101_{2}$, $-5_{10} = 1011_{2}$
\end{enumerate}

Некоторые из этих способов были реализованы. В 50--х и 60--х годах широко использовался четвертый способ. И специалисты теории информации разбились на два лагеря: те, кто использовал четвертый способ, и те, кто использовал пятый. Долго не могли примириться и компьютеры существовали и в том, и в другом виде. Это привело к тому, что в стандарте языка программирования Си не определено как именно представлять отрицательные числа. Если Вы не будете использовать стандартные конструкции языка (например, $a = b + c$), в которых язык Си сам считает значение из памяти и приведет к нужному отрицательному или положительному виду, а сразу обратитесь к внутреннему представлению памяти -- к ячейке по адресу (возьмете значение напрямую из ячейки), то хранящиеся там биты будут различны. Все будет зависеть от того, как работает Ваш компьютер: по четвертому способу или по пятому. Так же при складывании некоторых чисел, интерпретированных в обратном коде, в ограниченной разрядной сетке, возникает ошибка, и требуется корректировка результата. Пятый способ не требует корректировки. Поэтому было решено использовать для представления отрицательных чисел в ЭВМ дополнительный код.

\begin{center}
  \textbf{Алгоритм перевода двоичного числа в дополнительный код и из дополнительного кода в прямой}
\end{center}
\begin{enumerate}
\item Инвертировать все биты;
\item Прибавить единицу.
\end{enumerate}

Отличить, в каком коде представлено число в разрядной сетке -- в дополнительном или прямом, можно по старшему значащему биту. Если старший бит равен нулю -- число положительное, единице -- число отрицательное (например, в числе $1010_{2}$ старший бит равен $1$ -- число отрицательное и представлено в дополнительном коде).
\\\textbf{\emph{Важно! Нумерация бит в разрядной сетке начинается с нуля и идет справа налево.}}
\\
\\\emph{\textbf{Пример 6:}}
\\\emph{Задание:} перевести число $1101011011_{2}$ в дополнительный код.
\\\emph{Решение:} инвертируем биты: $1101011011_{2} \to 0010100100_{2}$;
\\Прибавляем единицу: $0010100100_{2} + 1_{2} = 0010100101_{2}$.
\\Ответ: $0010100101_{2}$.
\\
\\\emph{\textbf{Пример 7:}}
\\\emph{Задание:} представить число $-18_{10}$ в 8--разрядной сетке.
\\\emph{Решение:} так как у нас число отрицательное, то сначала переведем модуль данного числа в двоичную систему счисления: $|-18_{10}| = 18_{10} = 10010_{2}$.
Так как у нас 8--разрядная сетка, то дополним число незначащими нулями: $0001\ 0010_{2}$
Теперь представим его в дополнительном коде: $0001\ 0010_{2} \to 1110\ 1101_{2} + 1_{2} = 1110\ 1110_{2}$;\\
Ответ: $-18_{10} = 1110\ 1110_{2}$

\section{Диапазон значений}
Фиксированное значение разрядности хранимого числа определяет диапазон возможных значений, которые можно записать в отведенное количество байт.

Пусть в некотором компьютере переменная А хранится с использованием $k$ бит. Чтобы определить диапазон возможных значений, достаточно найти минимальное и максимальное значения А. 

\subsection{Беззнаковые числа}
Если мы рассматриваем только неотрицательные числа, то минимальным значением А будет 0. Это соответствует случаю, когда в каждом разряде числа А записан ноль.

Максимально представимым число А будет тогда, когда в каждый разряд записаны единицы. Это число, равное $2^k-1$.
Почему минус 1? Ответить на этот вопрос поможет простой пример: рассмотрим трехразрядное число в десятичной СС. Очевидно, что наибольшее такое число 999. Легко убедится, что число 999 можно получить, если вычесть единицу из минимального 4--разрядного числа ($1000 = 10^3$). Видим, что степень 10 соответствует количеству разрядов, которым будет ограничено представление числа. Аналогичное правило можно вывести и для двоичной СС. Тогда для расчёта диапазона представления целых неотрицательных чисел при наличии $k$--разрядной сетки компьютера можно применять следующую формулу: А $\in [0;2^k-1]$.

\subsection{Знаковые числа}
Так как при $k$--разрядном представлении отрицательного числа А в дополнительным коде старший разряд выделяется для хранения знака числа, то непосредственное значение числа А может храниться в $k-1$ разрядах. Поэтому для знаковых чисел при наличии $k$--разрядной сетки компьютера можно применять следующую формулу: А $\in [-2^{k-1};2^{k-1}]$.

\section{Флаги состояния процессора}

\textbf{Регистр флагов} -- регистр процессора, отражающий текущее состояние процессора.
\begin{table}[h]
\centering
\caption{Регистр флагов Intel x86}
\label{tab:Flags}
\begin{tabular}{|c|c|c|c|}
\hline
Номер & Обозна-- & \multirow{2}{*}{Название} & \multirow{2}{*}{Описание} \\
бита & чение & &
\\\hline
\multicolumn{4}{|c|}{FLAGS} \\
\hline
 0 & CF & Carry Flag & Флаг переноса
\\ 1 & -- & -- & Зарезервирован
\\ 2 & PF & Parity Flag & Флаг четности
\\ 3 & -- & -- & Зарезервирован
\\ 4 & AF & Auxiliary Carry Flag & Вспомогательный флаг переноса
\\ 5 & -- & -- & Зарезервирован
\\ 6 & ZF & Zero Flag & Флаг нуля
\\ 7 & SF & Sign Flag & Флаг знака
\\ 8 & TF & Trap Flag & Флаг трассировки
\\ 9 & IF & Interrupt Enable Flag & Флаг разрешения прерываний
\\ 10 & DF & Direction Flag & Флаг направления
\\ 11 & OF & Overflow Flag & Флаг переполнения
\\ 12 & \multirow{2}{*}{IOPL} & \multirow{2}{*}{I/O Privilege Level} & Уровень приоритета
\\ 13 & & &  ввода--вывода
\\ 14 & NT & Nested Task & Флаг вложенности задач
\\ 15 & -- & -- & Зарезервирован
\\\hline
\multicolumn{4}{|c|}{EFLAGS} \\
\hline
   16 & RF & Resume Flag & Флаг возобновления
\\ \multirow{2}{*}{17} & \multirow{2}{*}{VM} & Virtual--8086 & Режим виртуального
\\ & & Mode & процессора 8086
\\ 18 & AC & Alignment Check & Проверка выравнивания
\\ \multirow{2}{*}{19} &  \multirow{2}{*}{VIF} & Virtual & Виртуальный флаг
\\ & & Interrupt Flag & разрешения прерывания
\\ \multirow{2}{*}{20} & \multirow{2}{*}{VIP} & Virtual Interrupt & Ожидающее виртуальное
\\ & & Pending & прерывание
\\ \multirow{2}{*}{21} & \multirow{2}{*}{ID} & \multirow{2}{*}{ID Flag} & Проверка на доступность
\\ & & & инструкции CPUID
\\ 22 & \multirow{3}{*}{--} & \multirow{3}{*}{--} & \multirow{3}{*}{Зарезервированы}
\\ \dots & & &
\\ 31 & & &
\\\hline
\multicolumn{4}{|c|}{RFLAGS} \\
\hline
32 & \multirow{3}{*}{--} & \multirow{3}{*}{--} & \multirow{3}{*}{Зарезервированы}
\\ \dots & & &
\\ 63 & & & \\
\hline
\end{tabular}
\end{table}
\\После любой арифметической операции процессор автоматически без участия программиста заполняет регистр флагов состояния. Состояние процессора меняется после каждой арифметической операции.
\\Таблица \ref{tab:Flags} представлена для ознакомления на странице 39. Для курса <<Информатика>> необходимо знать следующие флаги:
\begin{enumerate}
  \item \textbf{CF (Carry Flag) -- Флаг переноса}. Устанавливается (принимает значение $1$) в случае, если происходит перенос за пределы разрядов или заем извне.
  \item \textbf{PF (Parity Flag) -- Флаг четности}. Устанавливается, если младший значащий байт результата содержит четное число единичных (ненулевых) бит. Изначально этот флаг был ориентирован на использование в коммуникационных программах: при передаче данных по линиям связи для контроля мог также передаваться бит четности.
  \item \textbf{AF (Auxiliary Carry Flag) -- Вспомогательный флаг переноса}. Устанавливается, если произошел заем или перенос между первым и вторым полубайтами (третьим и четвертым битами).
  \item \textbf{ZF (Zero Flag) -- Флаг нуля}. Устанавливается, если результат машинной операции по модулю 2 в степени $k$ (где $k$ -- разрядность ячейки) равен нулю (другими словами, принимает значение $1$, если результат выполнения операции равен нулю).
  \item \textbf{SF (Sign Flag) -- Флаг знака}. Устанавливается, если результат выполнения операции отрицателен (равен значению старшего значащего бита).
  \item \textbf{OF (Overflow Flag) -- Флаг переполнения}. Устанавливается, если в результате выполнения операции со знаковыми числами появляется одна из ошибок: при сложении положительных чисел получается отрицательный результат, или при сложении отрицательных чисел получается положительный результат.
\end{enumerate}
%\begin{center}
%  \textbf{\emph{Примеры для 16--разрядной сетки}}
%\end{center}

\emph{\textbf{Пример 1:}}
\\\emph{Задание:} сложить $14837_{10}$ и $5832_{10}$ в 16--разрядной сетке. Расставить флаги состояния процессора.
\\\emph{Решение:} для начала переведем исходные числа в двоичную систему счисления. $14837_{10} = 0011\ 1001\ 1111\ 0101_{2}$, $5832_{10} = 0001\ 0110\ 1100\ 1000_{2}$.
\\
\begin{minipage}[c]{10cm}
\begin{tabular}{r l c r | r r |}
\\
\hhline{~~~~----}
\multirow{2}{*}{+} & $0011\ 1001\ 1111\ 0101_{2}$ & = & $14837_{10}$ & CF = 0 & ZF = 0
\\ & $0001\ 0110\ 1100\ 1000_{2}$ & = & $5832_{10}$ &  PF = 1 & SF = 0
\\ \hhline{~--~~~}
 & $0101\ 0000\ 1011\ 1101_{2}$ & = & $20669_{10}$ & AF = 0 & OF = 0
\\\hhline{~~~~----}
\end{tabular}
\end{minipage}
\\
\\Так как $0101\ 0000\ 1011\ 1101_{2} = 20669_{10}$, то результат операции сложения в 16--разрядной сетке корректен.
\\
\\\emph{\textbf{Пример 2:}}
\\\emph{Задание:} сложить $21324_{10}$ и $13543_{10}$ в 16--разрядной сетке. Расставить флаги состояния процессора.
\\\emph{Решение:} для начала переведем исходные числа в двоичную систему счисления. $21324_{10} = 0101\ 0011\ 0100\ 1100_{2}$, $13543_{10} = 0011\ 0100\ 1110\ 0111_{2}$.
\\
\begin{minipage}[c]{10cm}
\begin{tabular}{r l c r | r r |}
\\
\hhline{~~~~----}
\multirow{2}{*}{+} & $0101\ 0011\ 0100\ 1100_{2}$ & = & $21324_{10}$ & CF = 0 & ZF = 0
\\ & $0011\ 0100\ 1110\ 0111_{2}$ & = & $13543_{10}$ &  PF = 0 & SF = 0
\\ \hhline{~--~~~}
 & $1000\ 1000\ 0011\ 0011_{2}$ & = & $-30669_{10}$ & AF = 0 & OF = 1
\\\hhline{~~~~----}
\end{tabular}
\end{minipage}
\\
\\Так как $1000\ 1000\ 0011\ 0011_{2} = -30669_{10} \ne 34867_{10} = 21324_{10} + 13543_{10}$, то результат операции сложения в 16--разрядной сетке некорректен. \\OF = 1: при складывании положительных чисел получили отрицательное.
\\
\\\emph{\textbf{Пример 3:}}
\\\emph{Задание:} аналогично примерам 1 и 2 -- сложить $-7453_{10}$ и $24732_{10}$.
\\\emph{Решение:} $24732_{10} = 0110\ 0000\ 1001\ 1100_{2}$, $-7453_{10} = 1110\ 0010\ 1110\ 0011_{2}$ ($-7453_{10}$ представляем в дополнительном коде для 16--разрядной сетки).
\\
\begin{minipage}[c]{10cm}
\begin{tabular}{r l c r | r r |}
\\
\hhline{~~~~----}
\multirow{2}{*}{+} & $0110\ 0000\ 1001\ 1100_{2}$ & = & $24732_{10}$ & CF = 1 & ZF = 0
\\ & $1110\ 0010\ 1110\ 0011_{2}$ & = & $-7453_{10}$ &  PF = 0 & SF = 0
\\ \hhline{----~~~}
 $1$& $0100\ 0011\ 0111\ 1111_{2}$ & = & $17279_{10}$ & AF = 0 & OF = 0
\\\hhline{~~~~----}
\end{tabular}
\end{minipage}
\\
\\Так как $0100\ 0011\ 0111\ 1111_{2} = 17279_{10}$, то результат операции сложения в 16--разрядной сетке корректен. Несмотря на то, что произошел выход за пределы разрядности сетки.